\documentclass[aspectratio=169]{beamer}
\usepackage[utf8]{inputenc} % utf8 file encoding
\usepackage[T1]{fontenc} % powerful pdf output encoding
% This file contains some settings , it need to be input by your tex file.

\usepackage[english]{babel}
\usefonttheme[onlymath]{serif}
\usepackage{xeCJK}
\usepackage{layout}

% TOC(table of content) using ordered list
\setbeamertemplate{section in toc}[sections numbered]
% unordered list using solid point
\setbeamertemplate{itemize item}{$\bullet$}
% set frame title for each page
\setbeamertemplate{frametitle}
{\vspace{0.5cm}
\insertframetitle
\vspace{-0.1cm}
}
% delete original navigation
\setbeamertemplate{navigation symbols}{}
% setfootline
\setbeamercolor{mycolor}{fg=black, bg=pink}
\setbeamertemplate{footline}{%
%\centerline{
\begin{beamercolorbox}[wd=\textwidth ,ht=1ex,dp=1ex,right,rightskip=0.5ex]{}%
    \tiny{\insertframenumber/\inserttotalframenumber}
\end{beamercolorbox}
%}
}


% define color
%\definecolor{alizarin}{rgb}{0.82, 0.1, 0.26} % a kind of red
%\definecolor{DarkFern}{HTML}{407428} % a kind of green
%\colorlet{main}{DarkFern!100!white} % first setting way(with color newly defined)
%\colorlet{main}{red!70!black} % second setting way(with internal color by setting gradient)
\definecolor{bistre}{rgb}{0.24, 0.17, 0.12} % a kind of black
\definecolor{mygrey}{rgb}{0.52, 0.52, 0.51} % a kind of grey
\colorlet{text}{bistre!100!white} % from now on, `text` implies the color defined
\colorlet{main}{green!50!black}

% set colors to elements, fg is the color itself, bg is the background color, !num! implies gradient
\setbeamercolor{title}{fg=main}
\setbeamercolor{frametitle}{fg=main}
\setbeamercolor{section in toc}{fg=text}
\setbeamercolor{normal text}{fg=text}
\setbeamercolor{block title}{fg=main,bg=mygrey!14!white}
\setbeamercolor{block body}{fg=black,bg=mygrey!10!white}
\setbeamercolor{qed symbol}{fg=main} % frame after proof
\setbeamercolor{math text}{fg=black}


%% use one of the two below
%\colorlet{main}{red!50!black}
\definecolor{dagrine}{RGB}{204,54,8}
\colorlet{main}{dagrine}

% \colorlet{main}{purple}

%-------------------main body-------------------------%

\author{Dwzb}
\title{Presentation Title}
\subtitle{sub Title}
\date{January 1, 2018}

\begin{document}

    \frame[plain]{\titlepage}

    \begin{frame}
        \frametitle{Outline}
        \tableofcontents
    \end{frame}

    \section{Page Title}

    \begin{frame}
        \frametitle{Page Title}

        \vspace{0.4cm}

        unordered list below

        \begin{itemize}
            \item The first item
            \item The second item
            \item The third item
            \item The fourth item
        \end{itemize}
        Hello, here is some text without a meaning. This text should show what a printed text will look like at this place. sin2(α) + cos2(β) = 1. If you read this text, you will get no information E = mc2. Really? Is there no information? Is there a difference between this text and some nonsense like “Huardest gefburn”? Kjift – not at all! A blind text like this gives you information about the selected font, how the letters are written and
        an impression of the look. √n a · √n b = √n ab. This text should contain all letters of the √na √a
        alphabet and it should be written in of the original language. √n = n . There is no bb
        need for special content, but the length of words should match the language. a √n b = √n a n b .

    \end{frame}

    \section{Display Theorem}

    \subsection{first subsection}

    \subsection{second subsection}



    \begin{frame}
        \frametitle{Display Theorem}
        \begin{theorem}
            $1 + 2 = 3$
        \end{theorem}
        \begin{proof}
            $$1 + 1 = 2$$
            $$1 + 1 + 1 = 3$$
        \end{proof}
    \end{frame}

    \section{Sample frame title}

    \begin{frame}
        \frametitle{Sample frame title}
        This is a text in second frame.
        For the sake of showing an example.

        \begin{itemize}
            \item Text visible on slide 1
            \item Text visible on slide 2
            \item Text visible on slide 3
        \end{itemize}

        \vspace{0.3cm}

        another example

        \begin{itemize}
            \itemsep0em
            \item Text visible on slide 1
            \item Text visible on slide 2
            \item Text visible on slide 3
        \end{itemize}
        \begin{exampleblock}{test}
            Test
        \end{exampleblock}

    \end{frame}

    \begin{frame}
        \frametitle{中文Titile}
        \begin{proof}
            $$
            \frac{1}{\displaystyle 1+
            \frac{1}{\displaystyle 2+
            \frac{1}{\displaystyle 3+x}}} +
            \frac{1}{1+\frac{1}{2+\frac{1}{3+x}}}
            $$
            $$\int_0^\infty e^{-x^2} dx=\frac{\sqrt{\pi}}{2}$$
            \begin{equation}
                x=y+3 \label{eq:xdef}
            \end{equation}
            In equation (\ref{eq:xdef}) we saw $\dots$
        \end{proof}
    \end{frame}
    \begin{frame}
        \frametitle{Also support 中文}
        Cool, you can use it in Chinese with out any modification.\par
        我和我的小伙伴们都惊呆了。
    \end{frame}
    \begin{frame}
        \begin{center}
            \LARGE{Thank you!}
        \end{center}
    \end{frame}
\end{document} 